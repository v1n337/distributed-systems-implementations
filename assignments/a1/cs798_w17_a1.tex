\documentclass[a4paper]{article}

%% Sets page size and margins
\usepackage[a4paper,top=3cm,bottom=2cm,left=3cm,right=3cm,marginparwidth=1.75cm]{geometry}
\usepackage{amsmath}
\usepackage{mathtools}
\usepackage{graphicx}
% \usepackage{todonotes}


\title{CS798 - Winter 2017 - Assignment 1}
\author{Vineet John (v2john@uwaterloo.ca)}
\date{}


\begin{document}
\maketitle

% \listoftodos

\part*{1. Raw Communication}

\section*{Q1. TCP - Sending 1 byte and receiving an ack}
\begin{center}
	\begin{tabular}{ |c|c| } 
		\hline
		& Round-Trip Time (Microseconds) \\ 
		\hline
		\hline
		Local & 52 \\ 
		\hline
		Remote & 688 \\ 
		\hline
	\end{tabular}
\end{center}

\section*{Q2. TCP - Sending large amounts of data}
\begin{center}
	\begin{tabular}{ |c|c| } 
		\hline
		& Throughput (MBps) \\ 
		\hline
		\hline
		Local & 126 \\ 
		\hline
		Remote & 1.36 \\ 
		\hline
	\end{tabular}
\end{center}

\section*{Q3. TCP - Overhead of connection establishment}
Experiment setup: 1KB transferred
\begin{center}
	\begin{tabular}{ |c|c| } 
		\hline
		& Latency (Microseconds) \\ 
		\hline
		\hline
		Connection-per-byte & 736052 \\ 
		\hline
		Batched (10B per connection) & 109027 \\ 
		\hline
	\end{tabular}
\end{center}
The overhead per extra connection attempt is \textbf{670 microseconds}.

\section*{Q4. UDP - Packet Loss} \label{udp_packet_loss}
\begin{center}
	\begin{tabular}{ |c|c| } 
		\hline
		& Data transferred before loss (bytes) \\ 
		\hline
		\hline
		Local & 32474296 \\ 
		\hline
		Remote & 6723152 \\ 
		\hline
	\end{tabular}
\end{center}

\newpage

\part*{2. Reliable UDP}

\section*{Q5. Reliable UDP - Throughput} \label{reliable_udp_throughput}
Experiment setup: 5GB transferred
\begin{center}
	\begin{tabular}{ |c|c|c| } 
		\hline
		& Packet Size (KB) & Throughput (MBps) \\ 
		\hline
		\hline
		Local & 50 & 1843 \\ 
		\hline
		Local & 5 & 296 \\ 
		\hline
		Remote & 50 & 10.75 \\ 
		\hline
		Remote & 5 & 6.29 \\ 
		\hline
	\end{tabular}
\end{center}

\section*{Q6. Reliable UDP - Bandwidth optimization}
From the experimental results shown in Q\ref{reliable_udp_throughput} we can conclude that the bandwidth can be optimized by making the packet size for each transfer as close to the threshold identified in Q\ref{udp_packet_loss}, as packets with higher sizes limit the connection overhead per transfer.

\newpage

\part*{3. Google RPC}

\section*{Q7. GRPC - Latency of request/response}
Experiment setup: 100 bytes as the transfer payload over 100 RPC calls. Average round-trip time measured.\\
The average request/response time was \textbf{256 microseconds}.

\section*{Q8. GRPC - Round trip time improvement}
Experiment setup: 100 bytes as the transfer payload over 100 RPC calls. Average round-trip time measured.\\
The first round-trip is approximately \textbf{4-5 times higher} than the overall average.

\section*{Q9. GRPC - Overhead compared to barebones UDP}
Experiment setup: 100 bytes as the transfer payload over 100 RPC calls/UDP transfers
\begin{center}
	\begin{tabular}{ |c|c| } 
		\hline
		& Latency (Microseconds) \\ 
		\hline
		\hline
		UDP & 16 \\ 
		\hline
		GRPC & 256 \\ 
		\hline
	\end{tabular}
\end{center}
The overhead for GRPC is \textbf{240 microseconds} for this payload, per RPC call.

\section*{Q10. GRPC - Throughput between remote machines}
Experiment setup: 100MB data to be transferred
\begin{center}
	\begin{tabular}{ |c|c| } 
		\hline
		& Throughput (MBps) \\ 
		\hline
		\hline
		Simple GRPC & 11.00 \\ 
		\hline
		Client-streaming GRPC & 11.74 \\ 
		\hline
	\end{tabular}
\end{center}

\end{document}
